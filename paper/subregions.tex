\section{Determination of the cartilage subregions}
\label{sec:Subregions}
As proposed by Wirth \& Eckstein, the two cartilage plates of the medial / lateral femoral condyle are divided into three subregions each (central, internal and external), while the two cartilage plates of the medial / lateral tibia are divided into five subregions each (central, internal, external, anterior, posterior). For nomenclature, refer to the table of acronyms below. 
\subsection{Determination of the femoral subregions}
For each plate of the femur, each subregion should encompass approximately 33\% of the plate volume. This was achieved in practice by splitting the volume into three parts along the x-axis; two splitting points $x1, x2$ are chosen arbitrarily, such that $x_{min} < x1 < x2 < x_{max}, x1 = \frac{x_{max} - x_{min}}{3}, x2 = 2 \cdot x1$, where $x_{max}, x_{min}$ are the maximum / minimum x values, respectively. $x1$ and $x2$ are then iteratively moved along the x-axis until the constraint is satisfied, i.e. 33\% of all data points lie left of $x1$, 33\% lie between $x1$ and $x2$ and 33\% lie right of $x2$.
\subsection{Determination of the tibial subregions}
For each plate, an ellipse around the center of the plate, aka the central region, should encompass approximately 20\% of the plate volume, and four triangles surrounding the ellipse should be of variable size. This was achieved in practice by first determining the center of gravity of the plate through K-Means clustering and constructing an ellipse around this point; a radius $r$ is chosen arbitrarily, and is lengthened iteratively until the constraint is satisfied, i.e. 20\% of all data points lie within the ellipse. The four triangles surrounding the ellipse are determined by calculating the corners $a, b, c, d$ of the plate, and data points are assigned to subregions according to their position relative to the vectors $\vec{ac}$ and $\vec{db}$, which is calculated via cross product.

\begin{minted}[escapeinside=~~,mathescape=true,breaklines,linenos]{text}
	Procedure to determine femoral subregions
	Given:
	x_axis := range of x-axis
	plate := data points (x, y, z) making up a cartilage plate
	
	Procedure:
	x_min, x_max := min(x_axis), max(x_axis)
	x_range := x_max - x_min
	x1 := x_range / 3
	x2 := 2 * x1
	first_third := empty set
	second_third := empty set
	
	while len(first_third) / len(plate) is not .33 do
		first_third := {d ~$\in$~ plate | d.x < x1}
		if len(first_third) > .33 
			x1 := x1 - 1
		else
			x1 := x1 + 1
	
	while len(second_plate) / len(plate) is not .33 do
		second_plate := {d ~$\in$~ plate | x1 < d.x < x2}
		if len(second_plate) > .33
			x2 := x2 - 1
		else
			x2 := x2 + 1
\end{minted}
\begin{minted}[escapeinside=~~,mathescape=true,breaklines,linenos]{text}
	Procedure to assign a femoral point to a subregion
	Given:
	plate := data points (x, y, z) making up a cartilage plate
	x1, x2 := split points
	
	for point in plate do
		if point.x < x1
			point.region = external/internal # depending on whether point lies in left or right plate
		if x1 < point.x < x2
			point.region = central
		else
			point.region = external/internal # depending on whether point lies in left or right plate
\end{minted}
\begin{minted}[escapeinside=~~,mathescape=true,breaklines,linenos]{text}
	Procedure to determine tibial subregions
	Given:
	plate := data points (x, y, z) making up a cartilage plate
	
	Procedure:
	r := 20
	c := KMeans(plate)
	points_in_ellipse := empty set
	
	while len(points_in_ellipse) / len(plate) is not .2 do
		points_in_ellipse := {d ~$\in$~ plate | dist(d, c) < r}
		if len(points_in_ellipse) > .2
			r = r / 2
		else
			r = r + .5
	
	x_min := {min(d.x) | d ~$\in$~ plate}
	x_max := {max(d.x) | d ~$\in$~ plate}
	y_min := {min(d.y) | d ~$\in$~ plate}
	y_max := {max(d.y) | d ~$\in$~ plate}
	
	a := (x_min, y_min)
	b := (x_max, y_min)
	c := (x_max, y_max)
	d := (x_min, y_max)
\end{minted}
\begin{minted}[escapeinside=~~,mathescape=true,breaklines,linenos]{text}
	Procedure to assign a tibial point to a subregion
	Given:
	plate := data points (x, y, z) making up a cartilage plate
	a, b, c, d := plate corners
	points_in_ellipse := set of points lying within the central ellipse
	
	Procedure:
	for point in plate do
		if point is in points_in_ellipse
			point.region := central
			
		ac := c - a
		db := b - d
		pc := c - point
		pb := b - point
		
		if ac ~$\times$~ pc > 0
			if db ~$\times$~ pc > 0
				point.region := internal
			else
				point.region := posterior
		else
			if db ~$\times$~ pc > 0
				point.region := anterior
			else
				point.region := external
\end{minted}