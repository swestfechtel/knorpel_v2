\section{Splitting of femoral cartilage}
\label{sec:Splitting}
As mentioned earlier, the femoral cartilage is split into multiple parts, namely the central weight-bearing zones, the anterior and the posterior sections (of the lateral and medial cartilage, respectively). The weight-bearing zones are defined as those regions of the cartilage which lie vis-a-vis, i.e. are in contact with, the central regions of the tibial cartilage (cLT/cMT, eLT/eMT, iLT/iMT). Once these are extracted, definition of the anterior and posterior parts is trivial: they are what is left of the cartilage on either side of the central weight-bearing zones (Refer to figure \ref{fig:femoral_subregions}). To circumvent the problem the convex nature of the posterior parts poses to mesh triangulation and function fitting, these sections are rotated by 90°. 
\par\noindent
Let the lateral and medial tibial cartilage plates be represented by two point clouds $LT$ and $MT$. Furthermore, let the central, external and internal subregions of the plates be represented by points clouds $cLT$, $eLT$, $iLT \subset LT$ and $cMT$, $eMT$, $iMT \subset MT$. For both point clouds, subsets $LT' \subset LT$ and $MT' \subset MT$ are defined such that
\newline
\begin{equation}
	LT' := \{\:(x,y,z)_i \in LT \:|\: x < \max_{x} cLT \land x > \min_{x} cLT\:\}
\end{equation}
\begin{equation}
	MT' := \{\:(x,y,z)_i \in MT \:|\: x < \max_{x} cMT \land x > \min_{x} cMT\:\}
\end{equation}
\newline
These subsets represent the parts of the tibial cartilage plates in contact with the femoral cartilage (in standing position). They can then be used to extract the corresponding parts of the femoral cartilage. Let the lateral and medial femoral cartilage regions be represented by two points clouds $LF$ and $MF$. For both point clouds, subsets $LF' \subset LF$ and $MF' \subset MF$ are defined such that
\newline
\begin{equation}
	LF' := \{\:(x,y,z)_i \in LF \:|\: x < \max_{x} LT' \land x > \min_{x} LT' \land y < \max_{y} LT' \land y > \min_{y} LT'\:\}
\end{equation}
\begin{equation}
	MF' := \{\:(x,y,z)_i \in MF \:|\: x < \max_{x} MT' \land x > \min_{x} MT' \land y < \max_{y} MT' \land y > \min_{y} MT'\:\}
\end{equation}
\newline
These subsets represent the parts of the femoral cartilage in contact with the tibial cartilage (in standing position), i.e. the central weight-bearing zones of the femoral cartilage.
\par\noindent
As mentioned before, definition of the remaining subregions of the femur is then trivial. Let the lateral and medial anterior and posterior regions of the femoral cartilage be represented by point clouds $aLF$, $pLF \subset LF$ and $aMF$, $pMF \subset MF$ such that
\newline
\begin{equation}
	aLF := \{\:(x,y,z)_i \in LF \:|\: x < \min_{x} LF'\:\}
\end{equation}
\begin{equation}
	aMF := \{\:(x,y,z)_i \in MF \:|\: x < \min_{x} MF'\:\}
\end{equation}
\begin{equation}
	pLF := LF \setminus (LF' \cup aLF)
\end{equation}
\begin{equation}
	pMF := MF \setminus (MF' \cup aMF)
\end{equation}
\newline
The subregions of the central weight-bearing zones are determined as suggested by \cite{wirth2008technique}.